\documentclass{assignment}

\usepackage{lipsum}

\newcommand*{\name}{}
\newcommand*{\id}{}
\newcommand*{\course}{Inferencia}
\newcommand*{\assignment}{Tarea 2}

% Entornos personales
\usepackage{amsmath,amssymb}

\newcommand{\Esp}{\mathbb{E}}
\newcommand{\Var}{\mathbb{V}ar}
\newcommand{\Prb}{\mathbb{P}}

\begin{document}

\maketitle


\section{Preguntas}
%%%%%%%%%%%%%%%%%%%% Pregunta 1 %%%%%%%%%%%%%%%%%%%% 
\subsection{ejercicio: Esperanza}
%%%%%%%%%%%%%%%%%%%%

\begin{enumerate}
	\item Considera la siguiente función de probabilidad.
\begin{center}
	\begin{tabular}{ |c|c|c|c|c| } 
 	\hline
 $x$ & $-1$ & $0$ & $1$ & $2$ \\ 
	 \hline
 $f(x)$      & 1/8 & 4/8 & 1/8 & 2/8 \\
 	\hline
	\end{tabular}
\end{center}
	Calcula la esperanza de la variable $X$ que tiene a  $f(x)$ como función de probabilidad. 

	
	\item Sea 
	\begin{align*}
	f(x)= \begin{cases} \text{ si } 2x \in (0,1) \\
		0 \text{ en otro caso.}
		\end{cases}
	\end{align*}
Calcula la esperanza de la variable $X$ que tiene a $f(x)$ como función de densidad.

	\item Considera a una variable aleatoria discreta con posibles valores en el conjunto $\{1,2, \dots n\}.$ Considera a esta variable $X$ distribuida de manera uniforme en este conjunto. Calcula la esperanza de $X$. 
	
	\item Considera la siguiente función de probabilidad.
\begin{center}
	\begin{tabular}{ |c|c|c|c|c|c| } 
 	\hline
 $x$ & $-2$ & $-1$ & $0$ & $1$ & $2$ \\ 
	 \hline
 $f_X(x)$ & 2/8 & 1/8 & 2/8 & 1/8  & 2/8 \\
 	\hline
	\end{tabular}
\end{center}		
	Para la función $h(x)= x^2$ considera la variable $Y=h(X)$. Da una tabla como la anterior para $f_Y$ y calcula la esperanza de $Y$. En el caso continuo, considera a $X$ una variable uniforme en el intervalo $(0,1)$, nuevamente sea $Y=h(X)$, calcula la esperanza de $Y.$
	 

	
\end{enumerate}


%%%%%%%%%%%%%%%%%%%% Pregunta 2 %%%%%%%%%%%%%%%%%%%% 
\subsection{ejercicio}
\begin{enumerate}
	\item Considera los puntos $(22,-22)$ y $(11,y)$ calcula el valor de $y$ tal que la pendiente de la recta que pasa por los puntos tiene pendiente $m=\pi.$
	\item De los puntos anteriores encuentra los valores de $a$ y $b$ de manera que al recta puede escribirse como $ax+b.$ 
	\item Calcula la recta perpendicular la recta de los incisos anteriores. 
	\item Investiga una ecuación para calcular la distancia de la recta al punto $(\pi,e).$
\end{enumerate}

%%%%%%%%%%%%%%%%%%%% Pregunta 3 %%%%%%%%%%%%%%%%%%%% 
\subsection{ejercicio: Intervalos de confianza}

\begin{enumerate}
	\item El departamento de tránsito de una cierta ciudad realizó una investigación para estimar el tiempo promedio $\mu,$ de reacción de un automovilista que conduce a una velocidad dada. Se hizo una estimación tomando una muestra aleatoria a 100 automovilistas donde se encontró que los tiempo de reacción de la muestra es  $\overline{x}=.86$ con $s^2=.09$. Comprobemos que los limites de confianza de la tabla. Además para cada uno de ellos calcule el intervalo $\overline{x}-\varepsilon < \mu < \overline{x}-\varepsilon$.

\begin{center}
\begin{tabular}{ |c|c|c|c|c| } 
 \hline
 Nivel de confianza & $90 \%$ & $95\%$ & $98\%$ & $99\%$ \\ 
 \hline
 $\frac{\varepsilon }{\sigma/\sqrt{n}}$      & 1.65    & 1.96   & 2.33   & 2.58  \\
 \hline
\end{tabular}
\end{center}


\item Supóngase que las estaturas de 100 estudiantes varones de una universidad representan una muestra aleatoria de las estaturas de los 1 546 estudiantes de esa universidad. Con $\overline{X}=67.45$ y $s^2=8.6136.$ (media muestral y desviación estándar muestral.) Calcula.

\begin{itemize}
    \item Los limites de confianza del $95\%$. 

    \item Los limites de confianza del $99\%$. 
\end{itemize}

	\item Para el caso de un nivel de confianza de $95\%$ el error es; $$1.96*\frac{.3}{\sqrt{100}}= .0588.$$ 
\begin{itemize}
    \item Para que el error se reduzca a la mitad ¿de que tamaño debe ser la muestra?
    \item y ¿para que se reduzca a la décima parte?
\end{itemize}

\end{enumerate}
%%%%%%%%%%%%%%%%%%%% Pregunta 4 %%%%%%%%%%%%%%%%%%%% 
\subsection{ejercicio: Método de máxima verosimilitud.}
%%%%%%%%%%%%%%%%%%%%
Investiga el método de máxima verosimilitud (likelihood) y encuentra el estimador de máxima verosimilitud para el parámetro $\lambda$ de una muestra $X_1, \dots, X_n$ distribuidas de manera exponencial.

\textbf{Definición:}
La función de verosimilitud de una muestra aleatoria $X_1, \cdots X_n$ denotada por $L(\theta)$ se define como la función de densidad conjunta, 

\begin{align*}
	L(\theta) = f_{X_1, \dots X_n}(x_1, \dots x_n;\theta) 
\end{align*}


%%%%%%%%%%%%%%%%%%%%%% Puntos extras %%%%%%%%%%%%%%%%%%%%
\section{Puntos extras}

%%%%%%%%%%%%%%%%%%%% Pregunta 1 %%%%%%%%%%%%%%%%%%%% 
 \subsection{Pregunta}
%%%%%%%%%%%%%%%%%%%%
Sea $X$ una variable con densidad dada por 
\begin{align*}
f(x) = \begin{cases}
\theta x ^{\theta-1} & \text{ si } 0 \leq x \leq 1 \\
0 & \text{ en otro caso.}
\end{cases}
\end{align*}
Encuentra un método para deducir al estimador $\theta = \frac{\overline{X}}{1-\overline{X}}$. ¿Es insesgado?

%%%%%%%%%%%%%%%%%%%% Pregunta 2 %%%%%%%%%%%%%%%%%%%% 
 \subsection{ejercicio}
Un intervalo de confianza al $(1- \alpha)*100\%$ para la media poblacional $\mu$ de una distribucón normal con varianza conocida $\sigma^2$ esta dada por 

\begin{align*}
	\Prb \left( -z_{\alpha/2} < \frac{\overline{X}-\mu}{\sigma /\sqrt{n}} < z_{\alpha/2} \right)=1-\alpha.
\end{align*}

calcula los limites de confianza y los intervalos de confianza. 

%%%%%%%%%%%%%%%%%%%% Pregunta 3 %%%%%%%%%%%%%%%%%%%% 
 \subsection{ejercicio}
%%%%%%%%%%%%%%%%%%%%
De un intervalo de confianza al $90\%$ para la media de una población con $\sigma = 5$ cuando se ha tomado una muestra de 25 cuya media muestral es 60. Es obligatorio usar el ejercicio 2 de esta sección, es decir, sustituir los valores encontrados en el ejercicio 2.  

%%%%%%%%%%%%%%%%%%%% Pregunta 4 %%%%%%%%%%%%%%%%%%%% 
\subsection{ejercicio}
%%%%%%%%%%%%%%%%%%%%
Sea $f$ una función diferenciable. Se sabe que la derivada de $f$ está relacionada con la recta tangente  a la gráfica de $f.$ ¿Qué relación existe entre la pendiente de recta tangente de $f'$ con $f$. Considera el criterio de la segunda derivada.
%%%%%%%%%%%%%%%%%%%% Pregunta 4 %%%%%%%%%%%%%%%%%%%% 
\subsection{ejercicio}
%%%%%%%%%%%%%%%%%%%%
El $n$-ésimo momento muestral esta dado por 

\begin{align*}
\frac{1}{n} \sum_ {i=0}^n X_i^k
\end{align*}
el método  de momentos para estimar un parámetro $\theta$ consiste en igualar los momentos muestrales con los momentos poblacionales. 

Sea $X_1 \cdots X_n$ una muestra distribuida $Ber(p)$ halle una estimación $\hat{p}$ para $p$ por medio del método de momentos. 


\section{Investigación: Teorema límite central y ley de los grandes números}

Son dos temas, el primero es el teorema limite central, en este ensayo procura evitar los cálculos y los detalles matemáticos avanzados. Explica las hipótesis del teorema, esto es ¿bajo que condiciones se cumple? ¿cuál es la importancia de este teorema? Procura no exceder mas de una cuartilla y agrega la bibliografía. 

La investigación de la ley de los grandes números tiene dos opciones, la ley débil y la fuerte. Considera este tema una investigación independiente de la anterior pero considera las instrucciones anteriores. 

Reflexiona, ¿cómo se usa la distribución normal para el cálculo de intervalos de confianza de muestras grandes? 


\end{document}
