\documentclass{assignment}

\usepackage{lipsum}

\newcommand*{\name}{}
\newcommand*{\id}{}
\newcommand*{\course}{Inferencia}
\newcommand*{\assignment}{Tarea 1}

\usepackage{amsmath,amssymb}

\newcommand{\Esp}{\mathbb{E}}
\newcommand{\Var}{\mathbb{V}ar}
\newcommand{\Prb}{\mathbb{P}}

\newcommand{\Nat}{\mathbb{N}}
\newcommand{\Rea}{\mathbb{R}}
\newcommand{\Rac}{\mathbb{Q}}
\newcommand{\Irr}{\mathbb{I}}

\begin{document}

\maketitle

\begin{enumerate}

%%%%%%%%%%%%%%%%%%%% Pregunta 1 %%%%%%%%%%%%%%%%%%%% 
\item \textbf{Pregunta 1: Estimadores}
%%%%%%%%%%%%%%%%%%%%

\begin{itemize}
    \item Sean $\hat{\Theta}_1$ y $\hat{\Theta}_2$ estimadores insesgados de un parámetro $\theta$. Sean $c_1+c_2=1$ constantes. ¿Es $c_1\hat{\Theta}+c_2\hat{\Theta}_2$ estimador? Si lo es, ¿de qué tipo?
    \item Sean $\hat{\Theta}_1$ y $\hat{\Theta}_2$ estimadores insesgados de un parámetro $\theta$. Sea $a$ una constante. ¿Es $a\hat{\Theta} + (1-a)\hat{\Theta}_2$ estimador? Si lo es, ¿de qué tipo?
    \item Sean $\hat{\Theta}_1$ y $\hat{\Theta}_2$ estimadores insesgados de un parámetro $\theta$. Sean $a$ y $b$ dos constantes. ¿Que valores deben ser $a$ y $b$ para que $a\hat{\Theta} + b\hat{\Theta}_2$ sea estimador insesgado? 
    \item Considera a $\overline{X} = 1/2 X_1 + 1/2 X_2$ y $\hat{Y} = 1/4 X_1 + 3/4 X_2$ estimadores de $\mu$. ¿Son sesgados? ¿Cuál es el estimador mas eficiente? 
    \item Si $\hat{\Theta}$ es un estimador consistente con $\theta$, ¿Que tan cierto es que $k\hat{\Theta}$ es consistente con $k\theta.$ 
    \item Sean $X_1, \dots, X_n$ una muestra aleatoria de la densidad 
    \begin{align*}
        f(x) = \theta f_1(x)+ (1-\theta)f_2(x)
    \end{align*}
    donde $\theta \in [0,1]$ y el valor esperado de $f_1(X)$ es $\mu_1$ y $\mu_2$ es el valor esperado de $f_2.$
    ¿Que tipo de estimador es?
    \begin{align*}
        \hat{\theta}= \frac{\overline{X}- \mu_2}{\mu_1-\mu_2}
    \end{align*}
\end{itemize}


%%%%%%%%%%%%%%%%%%%% Pregunta 2 %%%%%%%%%%%%%%%%%%%% 
\item \textbf{Pregunta 2: Muestras aleatorias}
Considera una definición alterna de estadístico. 

\textbf{Definición:}
Decimos que una estadística es una función de una muestra aleatoria $X_1, \dots, X_n$ tal que no depende de parámetros desconocidos y es una variable aleatoria.
%%%%%%%%%%%%%%%%%%%%
\begin{itemize}
    \item Considera a $X_1$, $X_2$ y $X_3$ una muestra aleatoria de una distribución $Ber(p)$. Encuentre la distribución, el valor esperado y la varianza de la estadística $X_1+X_2+X_3.$
    \item ¿Es $4(X_1 + \dots + X_n) - \mu$ un estadístico? ¿Es $\sigma^2+  X_1 + X_2$  un estadístico? 
\end{itemize}

%%%%%%%%%%%%%%%%%%%% Pregunta 3 %%%%%%%%%%%%%%%%%%%% 
\item \textbf{Pregunta 3:}
%%%%%%%%%%%%%%%%%%%%
Considere una urna con 3 bolas negras y 5 blancas. Se escoge una bola al azar, se registra su color y después se regresa a a urna. ¿Cuántas extracciones se necesitan realizar hasta obtener una bola negra por primera vez?

Una compañía aseguradora tiene una cartera particular de 550 pólizas contra robo de auto. La probabilidad de que cada una de estas pólizas presente una reclamación es de $p=.02$. Calcula la probabilidad de que un máximo de 10 pólizas presenten una reclamación. La segunda parte de este ejercicio es comparar con una distribución Poisson.  

%%%%%%%%%%%%%%%%%%%% Pregunta 4 %%%%%%%%%%%%%%%%%%%% 
\item \textbf{Pregunta 4:}
%%%%%%%%%%%%%%%%%%%%
Falso o verdadero.
\begin{enumerate}
    \item La esperanza de una variable aleatoria puede ser cero. 
    \item No hay dos variables aleatorias con la misma esperanza. 
    \item La esperanza de una variable aleatoria nunca es negativa. 
    \item La varianza de una variable aleatoria puede ser cero. 
    \item La varianza de una variable aleatoria nunca es negativa.
    \item No hay dos varianzas distintas con la misma varianza. 
    \item $\Esp(\Esp(X))=\Esp(X)$
    \item ¿Donde está el error?
\begin{align*}
    0 = \Var(0)= \Var(X-X)=\Var(X)+\Var(-X)=2 \Var(X).
\end{align*}
\end{enumerate}

\begin{center}
     \textbf{Puntos extras}
\end{center}
Cada uno de estos incisos tiene un valor de (+.2) al valor obtenido en los ejercicios.  

\begin{enumerate}
    \item Suponiendo que es igualmente probable que nazca un hombre (H)  o una mujer (M), y considerando la observación de 6 nacimientos. ¿Cuál de las siguientes es mas probable que ocurra (MHHMHM) o (MMMMHM) o (HMHMHM)?
    \item Sean $X_1$ y $X_2$ variables independientes distribuidas $Ber(p)$, calcula $P(X_1=0|X_1+X_2=1)$ y $P(X_1=1|X_1+X_2=1)$
    \item Calcule el número promedio de intentos necesarios para obtener un 6 en una sucesión de lanzamientos de un dado.  
    \item Sea $X$ una variable aleatoria $unif(a,b)$ tal que $\Esp(X)=4$ y $\Var(X)=3$ ¿cuáles son los valores de $a$ y $b$?
    \item Sea $X$ una variable $unif(0,1)$. Obtenga la distribución de la variable $Y=10X-5$.  
    \end{enumerate}

\end{enumerate}
\end{document}
