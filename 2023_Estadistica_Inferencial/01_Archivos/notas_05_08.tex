\documentclass{assignment}
\usepackage{lipsum}

\newcommand*{\name}{}
\newcommand*{\id}{}
\newcommand*{\course}{}

\newcommand*{\assignment}{Notas de clase: Lunes 08 de Mayo}

\usepackage{amsmath,amssymb}

\newcommand{\Esp}{\mathbb{E}}
\newcommand{\Var}{\mathbb{V}ar}
\newcommand{\Prb}{\mathbb{P}}

\newcommand{\Nat}{\mathbb{N}}
\newcommand{\Rea}{\mathbb{R}}
\newcommand{\Rac}{\mathbb{Q}}
\newcommand{\Irr}{\mathbb{I}}

\begin{document}

% Titulo
\maketitle


\section{Repaso matemáticas: Ejercicios lógicos}

\subsection*{Negación}
Dada una proposición o declaración $p$ la \textbf{negación} de $p$ se denota por $\neg q$ y es el valor de verdad contrario al de $p$.

\begin{table}[h]
    \centering
    \begin{tabular}{|c|c|}
    \hline
    $p$  & $\neg p$    \\
    \hline
    V  & F    \\
    \hline
    F  & V    \\
    \hline
    \end{tabular}
    \caption{Tabla de verdad de la negación}

\end{table}


\subsection*{Conjunción}
Dos proposiciones arbitrarias $p$ y $q$ que se combinan mediante la palabra \emph{y} para formar una nueva proposición cuyo valor de verdad depende de $p$ y $q.$ El valor de $p \wedge q$ es verdadero cuando $p$ como $q$ es verdadero. La tabla de verdad en este caso es,

\begin{table}[h]
    \centering
    \begin{tabular}{|c|c|c|}
    \hline
    $p$  & $q$ & $p \wedge q$   \\
    \hline
    V  & V & V   \\
    \hline
    V  & F & F   \\
    \hline
    F  & V & F   \\
    \hline
    F  & F & F   \\
    \hline
    \end{tabular}
    \caption{Tabla de verdad de la conjunción}

\end{table}

\subsection*{Disyunción}
Dos proposiciones arbitrarias $p$ y $q$ que se combinan mediante la palabra \emph{o} para formar una nueva proposición cuyo valor de verdad depende de $p$ y $q.$ El valor de $p \vee q$ es verdadero cuando alguna de las $p$ o $q$ es verdadera. La tabla de verdad en este caso es,

\begin{table}[h]
    \centering
    \begin{tabular}{|c|c|c|}
    \hline
    $p$  & $q$ & $p \vee q$   \\
    \hline
    V  & V & V   \\
    \hline
    V  & F & V   \\
    \hline
    F  & V & V   \\
    \hline
    F  & F & F   \\
    \hline
    \end{tabular}
    \caption{Tabla de verdad de la disyunción}
\end{table}

\subsection*{Ejercicio:}

\begin{table}[h]
    \centering
    \begin{tabular}{ |c|c|c|c|c|c|c| }
    \hline
    $p$  & $q$ & $\neg q$ & $\neg p$ & $p \vee \neg q$ &  $\neg (p \wedge \neg q)$ \\
    \hline
    V & V & & & & \\
    \hline
    V  & F & & & &\\
    \hline
    F  & V & & & &\\
    \hline
    F  & F & & & &\\
    \hline
    \end{tabular}

    \centering
    \begin{tabular}{ |c|c|c| }
    \hline
    $p$  & $q$ &  $\neg (p \vee \neg q)$ \\
    \hline
    V & V &   \\
    \hline
    V  & F &  \\
    \hline
    F  & V &  \\
    \hline
    F  & F &  \\
    \hline
    \end{tabular}
\end{table}


\section{Pruebas de hipótesis}
Una hipótesis es una afirmación que requiere ser puesta a prueba para determinar si es cierta o es falsa. Las pruebas que nosotros vamos a poner a prueba son mediante datos u observaciones. Esto con la idea de tomar mejores decisiones basadas en evidencias. 

\textbf{Recordatorio:} Sea $H_0$ el evento de lanzar un dado y registrar el valor de la cara del dado y sea $H_1$ el evento de lanzar consecutivamente 5 veces una moneda y registrar la cantidad de veces que sale cruz. 

Ahora queremos contestar preguntas del tipo, dado un número en el conjunto $\{0,1,2,3,4,5,6\}$ ¿cómo podemos determinar de que evento corresponde el número?


\begin{table}[h]
    \centering
    \begin{tabular}{|c|c|c|c|c|c|c|c|}
    \hline
       $x$  & 0 & 1 & 2 & 3 & 4 & 5 & 6   \\
       \hline
        $H_0:$ Dado  & 0  & 1/6 & 1/6 & 1/6 &1/6  & 1/6 & 1/6   \\ 
        \hline
        $H_1:$ Moneda  & 1/32 & 5/32 & 10/32 & 10/32 & 5/32 & 1/32 &  0 \\ 
        \hline
    \end{tabular}
    \caption{Tabla de probabilidades}

\end{table}

En general vamos a tener el esquema
\[
H_0: (\text{hipótesis nula}) \text{ vs } H_1: (\text{hipótesis alternativa})
\]

A $H_0$ es llamada hipótesis nula y es la que queremos poner a prueba o es la que es de nuestro interés. Mientras que $H_1$ es llamada hipótesis alternativa que nos sirve para poner a prueba a $H_0$ mediante probabilidades. 

\section{Definición:}
Una \textbf{hipótesis estadística} o simplemente hipótesis es una afirmación o conjetura acerca de la distribución de una o mas variables aleatorias. Una hipótesis es simple si especifica por completo la distribución de probabilidad, en caso contrario es llamada hipótesis compuesta. Una \textbf{prueba de hipótesis} es una regla para decidir si se acepta o se rechaza una hipótesis nula en contra de una hipótesis alternativa.  

\subsection*{Ejercicios}
De una hipótesis nula y una alternativa para los siguientes casos. 
\begin{enumerate}
	\item La cantidad de oficinas en el país que usan cierto software es de al menos $62\%$.
	\item Si el auto cobro de una tienda es mas rápida que un cajero atendido por una persona. 
\end{enumerate}

 
\section{Tipos de errores}
\begin{enumerate}
    \item El error de tipo I es cuando se rechaza a $H_0$ siendo esta verdadera.
    \item El error de tipo II es cuando no se rechaza a $H_0$ siendo esta falsa.
\end{enumerate}

De estos errores tenemos, la siguiente tabla. 

\begin{table}[h]
    \centering
    \begin{tabular}{|c|c|c|}
    \hline
    $p$  & $H_0$ cierta & $H_0$ falsa   \\
    \hline
    Rechazar $H_0$  & $\alpha$ & Decisión correcta   \\
    \hline
    Aceptar $H_0$ & Decisión correcta & $\beta$  \\
    \hline
    \end{tabular}
    \caption{Regiones críticas}
\end{table}

En este caso queremos $\alpha$ y $\beta$ como probabilidades que podemos usar para saber si bien estamos cometiendo un error de algún tipo. 

\subsection*{Ejemplo}
Supongamos que tenemos el lanzamiento de una moneda. Denotemos esto por 
\[
X_i = \begin{cases}
1 & \text{ si la moneda cae águila} \\
0 & \text{ si la moneda cae sol}.
\end{cases}
\] 
Tomemos 100 lanzamientos, o sea $X_1, \dots X_100$. Queremos inferir sobre $p$ la probabilidad del lanzamiento. 


\begin{align*}
\beta(\mu_1) & = \Prb \left( \text{No rechazar } H_0 \text{ cuando } \mu=\mu_1 \right) \\
& = \Prb \left( |Z|<z_{\alpha/2}| \mu=\mu_1 \right) \\
& = \Prb \left( \left|\frac{\overline{X}-\mu_0}{\sigma/\sqrt{n}} \right|<z_{\alpha/2}| \mu=\mu_1 \right) \\
& = \Prb \left(\mu_0 - z_{\alpha/2} \sigma/\sqrt{n}  < \overline{X} <  \mu_0 + z_{\alpha/2} \sigma/\sqrt{n}| \mu=\mu_1 \right) \\
& = \Prb \left(- z_{\alpha/2} + \frac{\mu_0-\mu_1}{\sigma/\sqrt{n}}  < \frac{\overline{X}-\mu_1}{\sigma/\sqrt{n}} <  z_{\alpha/2} + \frac{\mu_0-\mu_1}{\sigma/\sqrt{n}}\right) \\
& = \Phi \left( z_{\alpha/2} + \frac{\mu_0-\mu_1}{\sigma/\sqrt{n}} \right) - \Phi \left( -z_{\alpha/2} + \frac{\mu_0-\mu_1}{\sigma/\sqrt{n}}\right) \\
\end{align*} 
\begin{enumerate}
	\item ¿Cuál es la distribución de $\frac{1}{n}\sum_{i=1}^{100} X_i$.
	\item ¿Cuál es el valor esperado?
	\item Si $\hat{p}=\frac{1}{n}\sum_{i=1}^{100} X_i$ ¿A que parámetro y que tipo de estimador es?
\end{enumerate}

\section*{Ejercicios}
Consideremos $X_1, \dots X_{30}$ variables independientes distribuidas $Norm(\theta,1)$. Encuentre el nivel de confianza de los intervalos.

\begin{enumerate}
	\item $\left[\hat{\Theta}-\frac{2.14}{\sqrt{30}},\hat{\Theta}+\frac{2.14}{\sqrt{30}} \right]$.
	\item $\left[\hat{\Theta}-\frac{1.85 \sigma}{\sqrt{N}},\hat{\Theta}+\frac{1.85 \sigma}{\sqrt{N}} \right]$.
\end{enumerate}

Sea $X$ una $Norm(\mu,\sigma)$. Para $N=15$ se encontró que,

\[
\sum_{n=1}^N X_n = 250
\]

\[
\sum_{n=1}^N X_n^2 = 10,000
\]

encuentre el intervalo de confianza del $95\%$ para $\mu.$

Sea $\hat{\theta}$ la media muestral de un conjunto de $N$ datos si se sabe que provienen de una $Norm(\theta,3^2)$, encuentre $N$ tal que 
\[
\Prb(\hat{\Theta}-1<\theta < \hat{\Theta}+1)=.95
\] 
\section{Valor critico}

\section{$p$-valor}

\
\end{document}
Tenemos dos términos importantes, el

\begin{enumerate}
	\item 
\end{enumerate}
