\documentclass{assignment}

\usepackage{lipsum}

\newcommand*{\name}{}
\newcommand*{\id}{}
\newcommand*{\course}{}


\newcommand*{\assignment}{Notas de clase: Jueves 11 de Mayo}

\usepackage{amsmath,amssymb}

\newcommand{\Esp}{\mathbb{E}}
\newcommand{\Var}{\mathbb{V}ar}
\newcommand{\Prb}{\mathbb{P}}

\newcommand{\Nat}{\mathbb{N}}
\newcommand{\Rea}{\mathbb{R}}
\newcommand{\Rac}{\mathbb{Q}}
\newcommand{\Irr}{\mathbb{I}}

\begin{document}

\maketitle
\tableofcontents
\newpage

\section{Repaso matemático: Transformaciones invertibles}
Decimos que una función $f: A \to B$ es \textbf{invertible} si existe una función $g:B \to B$ de manera que $f(g(b))=b$ y $g(f(a))=a$ para los elementos $b$ en $B$ y $a$ en $A$. 

\begin{enumerate}
    \item $f(x)=x^3$ y $g(y)= \sqrt[3]{y}$,
    \item $f(x)= \sin(x)$ y $g(y)= \arcsin(y)$
\end{enumerate}

¿Son las funciones $f(x)=5x-3$, $g(y)=\frac{y+3}{5}$ funciones inversas?


\subsection*{Teorema de probabilidad total}
Un espacios muestral $\Omega$. Una colección de conjuntos $B_1, \dots , B_m$ es una partición de $\Omega$ si 
\begin{enumerate}
	\item $B_i\neq \emptyset$ para toda $i.$
	\item $B_i \cap B_j = \emptyset$ para toda $i.$
	\item $\cup_{i=1}^n B_i = \Omega$.
\end{enumerate}

El teorema de probabilidad total se enuncia de la siguiente manera, para todo conjunto $A$, se cumple que,

\[
\Prb(A)=\sum_{i=1}^n \Prb(A|B_i) \Prb(B_i)
\]

\subsection*{Ejemplos:}
 
\begin{enumerate}
	\item Una urna contiene 3 bolas blancas y 4 bolas negras. Se extraen dos bolas al azar una después de otra y sin reemplazo. Calcula la probabilidad de que,
	
	\begin{itemize}
		\item la segunda bola sea negra dado que la primera bola fue negra. 
		\item la segunda bola sea del mismo color que la primera.
		 \item la segunda bola sea blanca
		 \item la primera bola sea blanca dado que la segunda fue blanca. 
		 
	\end{itemize}
	
	\item Un estudiante contesta un examen de opción múltiple, cada pregunta tiene 4 respuestas. Si el estudiante sabe la pregunta contesta la opción correcta. Si no escoge al azar con una probabilidad de .6 de acertar. 
	
	\begin{enumerate}
		\item Calcula la probabilidad  de que el estudiante tenga correcta una pregunta al azar. 
		\item si el estudiantes obtuvo la respuesta correcta a una pregunta al azar. ¿Cuál es la probabilidad de que haya sabido verdaderamente la respuesta? 
	\end{enumerate}
	
	\item Una urna contiene 4 bolas blancas y 6 azules. Se lanza un dado y se extraen bolas, tantas como se indica el dado. Considere la extracción sin orden y sin remplazo. Encuentra la probabilidad de que las bolas sean blancas. 
	\item El problema de Monty Hall.
	
\end{enumerate}



\section{Teorema de Bayes}
Sea $A$ un evento con $\Prb(A)>0.$ Se cumple que,

\[
\Prb(B_j|A) = \frac{\Prb(A|B_j)\Prb(B_j)}{\sum_{i=1}^n\Prb(A|B_j)\Prb(B_j)}
\]

\subsection*{Ejemplo}
En un laboratorio se desarrolla una prueba para detectar una enfermedad. Sean $Enf$ el evento \emph{el individuo está enfermo} y $Neg$ el evento \emph{la prueba es negativa}. Sobre la eficacia de esta prueba se sabe
\begin{enumerate}
	\item $\Prb (Neg^c | Enf)=.95$
	\item $\Prb (Neg | Efn^c)=.96$
	\item $\Prb (Efn)=.01$
\end{enumerate}

\begin{itemize}
\item ¿Qué significarían los siguientes eventos $\Prb(Efn|Neg)$ y $\Prb(Efn|Neg^c)$?
\item ¿Qué significarían los siguientes eventos $\Prb(Efn^c|Neg^c)$ y $\Prb(Efn^c|Neg)$?
\end{itemize}



\section{Independencia de eventos}
Decimos que dos eventos $A$, $B \subset \Omega$ son independientes si se cumple que 

\[
\Prb(A \cap B) = P(A)*P(B)
\]

\textbf{Observación:}
Ser independientes no implica que sean ajenos. De la misma manera dos eventos ajenos no implica que sean independientes. 

\subsection*{Ejemplos:}

\begin{enumerate}
\item El lanzamiento de una moneda 3 veces. Tiene como espacio muestral,

\[
\Omega=\{aaa, aas, asa, saa, ass, sas, ssa, sss\}.
\]

\begin{enumerate}
	 \item $A:$ \emph{Caen a lo mas 2 águilas.}
	 \item $B:$ \emph{Caen al menos 2 águilas.}
	 \item $C:$ \emph{Caen 3 águilas o 3 soles.}
\end{enumerate}

\item Supongamos que las probabilidades de que una familia tenga un hijo o tenga una hija son iguales. Si la familia tiene dos hijos, considere los siguientes eventos. 

\begin{enumerate}
	\item $A:$ \emph{la familia tiene hijos de ambos sexos.}
	\item y $B:$ \emph{a lo mas uno de los hijos es varón.}
\end{enumerate}


	\item Supongamos que al lanzar una moneda \emph{honesta} obtienen 5 águilas en sucesión. ¿Cuál es la probabilidad de que en el sexto lanzamiento también sea águila?


\end{enumerate}


\subsection*{Ejercicios:}
Los ejercicios son (+.25) para la tarea 4.
\begin{enumerate}
\item ¿son independientes dos a dos?
\begin{itemize}
	\item Sea $\Omega = \{ 1,2,3,4,5,6,7,8 \}$ y los eventos,
	\item $ A = \{ 1,2,3,4  \}$
	\item $ B = \{ 1,5,6,7  \}$
	\item $ C = \{ 1,2,3,5  \}$
\end{itemize}



\item Se lanza un dado equilibrado dos veces. Determine si los siguientes pares de eventos son independientes. 

\begin{itemize}
	\item $A:$ \emph{la suma de los dados es 6.}
	\item y $B:$ \emph{el primer resultado es 4.}
\end{itemize}

\begin{itemize}
	\item $A:$ \emph{la suma de los dados es 7.}
	\item y $B:$ \emph{el segundo resultado es 4.}
\end{itemize}

R= Los dos primeros no, los dos segundos si.

\item Sean $C$ y $D$ dos eventos independientes, tales que $\Prb(C)=c$ y $\Prb(D)=d$, Calcula las probabilidades,

	\begin{enumerate}
		\item ninguno de estos dos eventos ocurra,
		\item solamente uno de los eventos ocurra,
		\item al menos uno de los eventos ocurra,
		\item los dos eventos ocurren,
		\item a lo mas uno de los eventos pasa
	\end{enumerate}
	\item Deduzca la ecuación 
	\[ 
	\Prb(A_1 \cup A_2 \cup A_3)=1-\Prb(A_1^c)*\Prb(A_2^c)*\Prb(A_3^c)
	\]
\end{enumerate}

\section{Probabilidad condicional}
Sean $A$ y $B$ dos eventos. Consideremos el caso cuando $\Prb(B)>0.$ \emph{la probabilidad condicional del evento $A$ dado el evento $B$} se define como,
\[
\Prb(A|B)=\frac{\Prb(A \cap B)}{\Prb(B)}.
\]


\subsection*{Ejercicios:}
Los ejercicios son (+.25) para la tarea 4.
\begin{enumerate}
	\item Sea $\Prb(A)=.5$ y $\Prb(A \cup B)=.6$, encuentre $\Prb(B)$ en cada caso,
	\begin{enumerate}
  	  \item $A$ y $B$ son ajenos.
  	  \item $A$ y $B$ son independientes. 
  	  \item $\Prb(A|B)=.4$
	\end{enumerate}

\item ¿Cierto o falso?
		\begin{enumerate}
		\item $\Prb(A) < \Prb(A|B)$
		\item $\Prb(A) = \Prb(A|B)$
		\item $\Prb(A) > \Prb(A|B)$
		\end{enumerate}
\item Crea un ejemplo donde se cumpla
		\begin{enumerate}
		\item Con $\Prb(A|B)=0$ $\Prb(A)>0$ 
		\item $\Prb(A|B^c) = \Prb(A^c|B)$
		\end{enumerate}
		
\item ¿Qué significarían los siguientes eventos $\Prb(Efn^c|Neg^c)$ y $\Prb(Efn^c|Neg)$?
\end{enumerate}
\end{document}

